\section{Discussion \& Conclusion}

\enquote{It would seem, therefore,
that the influence of the longwave emitted radiation would be almost
indistinguishable from a small change in the gravitational constant of the Earth, for
low eccentricity orbits.
As a consequence, one would expect the shortwave
component to have a greater orbital effect than the longwave component, in spite of
the comparable magnitudes of their accelerations.}~\cite{Knocke1989}


recommendations on which models to use


Future work:
\begin{itemize}
    \item account for moon topography for occlusion~\cite{Mazarico2018}, could otherwise lead to large misrepresentation of eclipses for $\beta > \ang{70}$
    \item Self-shadowing, particularly for SRP, can reduce effective cross section by up to 40\%~\cite{Mazarico2018}
    \item fit new SH albedo model or just use map
    \item accurate thermal reradiation (e.g. \cite{Marshall1994})
    \item account for non-diffuse reflection of lunar surface, i.e.opposition effect due to shadow hiding and coherent backscatter, which can greatly increase irradiance at small phase angles~\cite{Buratti1996} (this would only be relevant at beta = 0 since low phase angles do not occur for large betas; phase angle is low if target is above subsolar point); a map of Hapke parameters exists~\cite{Sato2014}
\end{itemize}


