\section{Conclusion}

We described a collection of \gls{RP} models of varying levels of complexity, then examined the differences in short-term orbital effects of \gls{RP} on \gls{LRO} between these models. There are large seasonal differences in the \gls{RP} accelerations: for small $\beta$ (e.g., around September), the accelerations are mainly radial and along-track, while they are predominantly cross-track for $\beta \approx \pm \ang{90}$ (e.g., around June). After 2.5 days, the position diverged from the no-\gls{RP} baseline by \qty{1100}{\m} in June and \qty{80}{\m} in September. Periodic variations of up to \qty{40}{\m} are superimposed on the secular differences over one orbital revolution. In September, the periodic variations are damped by lunar accelerations that oppose solar accelerations. Large differences also exist between the representations of \gls{LRO} as a cannonball and a paneled target: due to the cannonball's symmetry, accelerations are more uniform and generally smaller than those of a paneled target, which can have the solar array track the Sun. Asymmetric effects, which a cannonball cannot represent, can lead to qualitative differences. Thermal radiation dominates the lunar emissions, and a constant albedo distribution is both sufficiently accurate and computationally cheaper than the spherical harmonics expansion \gls{DLAM1}.

Our results showed that \gls{RP} is essential for precise orbit determination. Both the total and radial accuracy requirements of \gls{LRO} would be violated otherwise. However, not all models are worth the computational effort. We recommend the following setup:
\begin{itemize}
    \item Solar radiation should be included since it is significant yet computationally cheap.
    \item Lunar thermal and constant albedo radiation should be included since it only increases walltime duration by \qty{20}{\percent} and affects secular and periodic variations significantly. To reduce the performance impact, fewer rings could be used, although the lunar irradiance may then be underestimated.
    \item The spatial variations in albedo from \gls{DLAM1} do not increase the accuracy much despite the performance penalty of \qty{80}{\percent}. Therefore, it should not be included unless the utmost accuracy is desired (particularly in September). The spherical harmonics expansion could also be truncated to improve performance.
    \item The paneled target should be included since the cannonball underestimates accelerations and does not account for Sun tracking of the solar array. The performance impact of the paneled target is negligible.
    \item The cannonball target should only be included if its coefficient is estimated since a constant coefficient cannot represent changes in geometry and orientation. Even then, consistent estimation of the coefficient is difficult at small $\beta$~\cite{Slojkowski2014}.
\end{itemize}

While we restricted our investigation to a small number of relatively simple models, the short-term orbital effect of these more involved models should also be investigated:
\begin{itemize}
    \item Self-shadowing, particularly for solar radiation, can reduce the effective cross-section by up to \qty{40}{\percent} for large $\beta$~\cite{Mazarico2018}.
    \item Moon topography can advance eclipse onset by up to \qty{480}{\s} for $\beta > \ang{70}$~\cite{Mazarico2018}. The conical shadow model should be replaced by one that evaluates lines of sight based on topography.
    \item \gls{DLAM1} was published in 1999 based on miscalibrated Clementine imagery, which overestimates the actual albedo. A new spherical harmonics model should be fitted from more recent, properly calibrated imagery.
    \item Accelerations due to thermal reradiation by the spacecraft itself can be significant. Instantaneous reradiation should be replaced by a model that accounts for heating and conduction.
    \item The lunar opposition effect increases albedo radiation for low phase angles much more than Lambertian reflectance predicts. Such phase angles occur for small $\beta$. The Hapke \gls{BRDF} with spatially resolved parameter maps~\cite{Sato2014} should be used for albedo reflection.
    \item Post-sunset lunar thermal radiation is underestimated because gradual cooling to nighttime temperatures is not reflected in our thermal model. It should be replaced with a more physical model due to the large magnitude of thermal compared to albedo radiation.
\end{itemize}



\FloatBarrier

