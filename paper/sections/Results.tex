\section{Results}

beta angle slightly less than 90 degrees leads to sinusoidal acceleration

IAU\_MOON is in worst case 155 m off~\cite{NAIF2020} (Special PCK and FK for Earth and Moon, slide 14)

for radiation pressure acceleration, add partial/full eclipse to time axis

absolute acceleration magnitude likely influenced by mass uncertainty
rp acceleration magnitude increases as mass decreases

plots for beta = 80 (constant sunlight, Jun 26) and beta = 0 (orbital plane intersects sun, Sep 26)

start at 2010-06-26 06:00:00 UTC, which corresponds to 2010-06-26 06:01:06 TDB

effect of instantaneous reradiation

static vs dynamic performance
static number of panels
dynamic number of panels per ring

variation in altitude is in part due to assumption of spherical moon (polar radius is 2.1 km less than equatorial)

static vs dynamic have different types of inaccuracy
dynamic at low resolution has bias and converges globally towards accurate version
static at low resolution looks like noise and noise is reduced when increasing resolution

arc length 2.5 days, which is also used for LRO orbit determination~\cite{Mazarico2011}

neglecting self-shadowing overestimates area~\cite{Mazarico2009}, but minimal self-shadowing in most cases for LRO~\cite{Slojkowski2015}

Thermal radiation may cause an offset of 1-2 meters over an arclength of 2.5 days~\cite{Bauer2016}

our maximum eclipse time of 48 min agrees with \cite{Tooley2010}

kink in cross-track SRP also seen in SELENE \cite{Kubooka1999}, search for explanation
