\section{Results}

no knowledge of true RP accelerations
therefore, compare to baseline

Also use ~\cite{Borderies1990} as reference for plots and discussion, especially about relation of acc and change in elements

\subsection{Simulation setups}

Solar with/without
Lunar with/without
Albedo constant/dlam
LRO cannonball/paneled
Occultation with/without (for solar only?)


\subsection{Accelerations}
thermal vs albedo

kink in cross-track SRP also seen in SELENE \cite{Kubooka1999}, search for explanation

Variation with orbital position and time of year (correlate with relative sun position and albedo map)

beta angle slightly less than 90 degrees leads to sinusoidal acceleration

show partial/full eclipse on time axis

absolute acceleration magnitude influenced by mass uncertainty
rp acceleration magnitude increases as mass decreases
17\% higher mass at start -> 17 \% lower acceleration magnitude

angle-based thermal behaves quite similar to albedo, but does not vanish in eclipse



\subsection{Change in orbital element}

compare wih Gauss perturbing equations (analytical solution to change of osculating elements based on accelerations), e.g. ~\cite[Sec.~3.2]{Lucchesi2006}


\subsection{Performance}
no special setup like cpu pinning or disabled hyperthreading for benchmarking
only on one setup
Performance may vary in other situations~\cite{Mytkowicz2009}
still, a good indication
also mention minimum

albedo model can increase computational demand by several hundred pct \cite{Nicholson2010}

