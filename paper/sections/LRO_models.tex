\section{Radiation pressure modeling for LRO}

\subsection{Lunar radiation pressure}
use 5 rings for moon due to convergence analysis and results from~\cite{Floberghagen1999}
\cite{Nicholson2010} also uses 5 rings for LRO
Need more with DLAM-1 than knocke due to higher frequency albedo dist or lower altitude?

%Albedo

no seasonal or diurnal albedo variation on Moon, as opposed to Earth~\cite{Knocke1989}

Knocke argues that Earth can be reasonably represented using diffuse albedo reflection only
Is this the case also for moon?

find source for how much lunar albedo varies with viewing angle, i.e. if the moon albedo would benefit from an angular distribution model

albedo value used should be for broadband shortwave (\SIrange{0.2}{4}{\micro\meter}, peak at \SI{0.4}{\micro\meter})~\cite{Knocke1989}, which accounts for most of solar radiation
albedo used for moon is 0.19 (750 nm, which corresponds to maximum reflectivity~\cite{Floberghagen1999}), which is mean of DLAM-1, even though 0.12 is commonly cited
DLAM-1 is derived from clementine imagery, which is known to overestimate albedo~\cite{Shkuratov2011}
This is to enable better comparison, but if a constant albedo value were used, this amounts to linear scaling


% Thermal

use angle-based model from Lemoine
Flux from Lemoine agrees with \cite[Table~8]{Tooley2010}

Constant-emission model from Knocke is not appropriate for moon since it gets very cold -> Knocke would result in constant emission (only varies by XX\% due to change in Moon--Sun distance), will not be investigated further here

\subsection{LRO target}

to find cannonball area and coefficient, some authors use raytracing~\cite{Hattori2019}, we just use weighted average
finding a single rp coefficien is virtually impossible since it changes~\cite[p~580]{Vallado2013}

Different values for A and Cr in literature:
\cite{Nicholson2010}: 14, 1.0 (for daily/not precision OD, no changing orientation, solar only) --> use this one
\cite{Bauer2016}: 10, 1.2 (no changing orientation, solar only)
\cite{Slojkowski2015}: first 1.67, then 0.96 after estimation
\cite{Mazarico2018}: 1.03 +- 0.24 (1.04 in sep, 1.4 in jun, but rather a scale factor for paneling than cannonball coefficient)

mass at start of science orbit (15 Sep 2009): 1271.9 kg
mass at end of science orbit (11 Dec 2011): 1087.0 kg
use end of science orbit mass for all scenarios to get worst case scenario
fixed mass to enable comparison
also show mass history

effect of self-shadowing on LRO orbit is small~\cite{Loecher2018}
neglecting self-shadowing overestimates area~\cite{Mazarico2009}, but minimal self-shadowing in most cases for LRO~\cite{Slojkowski2015}

instantaneous reradiation
will not be investigated further
describe how results change (simple scaling?)
Thermal radiation may cause an offset of 1-2 meters over an arclength of 2.5 days~\cite{Bauer2016}



\subsection{LRO orbit geometry}
variation in altitude is in part due to assumption of spherical moon (polar radius is 2.1 km less than equatorial) -> leads to change in lunar RP magnitude over orbit
sun beta over year + eclipse periods

our maximum eclipse time of 48 min agrees with \cite{Tooley2010}



\subsection{Simulation setup}

earth albedo + thermal radiation can be neglected for LRO since it is less than 0.1\% of solar radiation at moon

solar array tracks Sun, HGA tracks Earth~\cite{Tooley2010}
start at start at 26 June 2010 06:00:00
Earth eclipses Sun during this time
Moon does not eclipse Sun (Sun beta angle is about -90 deg, see~\cite{Tooley2010})


Operational LRO OD does not use lunar albedo due to computational demand, but used for offline reprocessing.
Self-shadowing from \citeauthor{Mazarico2009} is used for reprocessing~\cite{Nicholson2010}

arc length 2.5 days, which is also used for LRO orbit determination~\cite{Mazarico2011}
step 5 s, which is also used for LRO orbit determination~\cite{Mazarico2018}

MOON\_PA frame, IAU\_MOON is in worst case 155 m off~\cite{NAIF2020} (Special PCK and FK for Earth and Moon, slide 14)

integrator + propagator params


maybe show figure from poster


two arcs, one for beta = 0 and beta = 90